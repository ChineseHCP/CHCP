\section{Modelling and plotting parametric responses}

In addition to analysing the data in the above "categorical" design, we now illustrate an alternative "parametric" design in which repetition is viewed as a continuum. In this case, the response to each presentation of famous and non-famous faces is modulated by the time interval (lag) since the previous presentation of that face (for first presentations, this lag is infinite). 

\subsection{Specification and estimation}

Finally, consider the effects of the interval between first and second presentations of famous and non-famous faces. For this purpose, an alternative statistical model needs to be estimated, with four trial types. (1) First and second presentation of a non-famous face (N1 and N2 collapsed). (2) First and second presentation of a famous face (F1 and F2 collapsed). (3) Errors in judging non-famous faces. (4) Errors in judging famous faces. These are modelled with the canonical hrf alone, with two parametric (exponential) modulations added for second presentations of non-famous and famous faces. See the README.txt for details of design specification.

\subsection{Inference and plotting}

After completion of model estimation, press `Results' and select the SPM.mat file. When the Contrast Manager appears, define an F-contrast `Effect of Lag (on canonical N+F)' (name) and `0 1 0 1', and a t-contrast `Canonical: Faces $>$ Baseline' (name) and `1 0 1 0'. Select the F-contrast, specify `mask with other contrasts' (yes), select `Canonical: Faces $>$ Baseline', specify `uncorrected mask p-value' (accept default), `nature of mask (inclusive), `title for comparison' (accept default), `corrected height threshold' (no), and `corrected p-value' (accept default). When the MIP appears, press `Volume'.

The table displays all clusters where the exponential change in activation between first and second presentations for famous OR non-famous faces for the canonical hrf is significantly different from zero AND the canonical hrf for the first two conditions is significantly larger than zero (baseline). We can now plot these parametric effects at particular voxels.

To plot these parametric effects, select the R fusiform region (45 -60 -18, similar to the region identified in the previous categorical analysis for repetition effects), and press 'plot'. Select 'Plots of parametric responses' and select 'F' (or 'N').  Note that the 'attrib' option does not allow adjustment of the scale of the Z-axis; therefore, to change the scale for all axes, type e.g. 'figure (1), axis ([0 30 0 100 0 1])' in the Matlab window.

Note that these repetition effects are transient (decrease with increasing intervals between first and second presentations), especially for famous faces.
